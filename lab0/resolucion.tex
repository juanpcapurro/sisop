\documentclass[a4paper]{article}
\usepackage[utf8]{inputenc}
\usepackage[width=1.00cm, height=0.00cm, left=2.00cm, right=2.00cm, top=2.00cm, bottom=2.00cm]{geometry}
\usepackage{amsmath}
\usepackage{amsfonts}
\usepackage{amssymb}
\usepackage{graphicx}
\usepackage{listings}
\usepackage{multirow}
\usepackage{url}
\usepackage{tabularx}
\usepackage{hyperref}
\usepackage{fancyhdr}
\usepackage{authblk}
\usepackage[T1]{fontenc}
\usepackage{ascii}
\usepackage{textcomp}
\usepackage{eurosym}
\usepackage{subcaption}
\usepackage[absolute]{textpos}

\usepackage{tikz,amsmath, amssymb,bm,color}
\usetikzlibrary{decorations.text}
\usetikzlibrary{decorations.pathreplacing}
\usetikzlibrary{snakes,matrix,positioning,backgrounds,fit,shapes.geometric}
\usetikzlibrary{patterns}
\usetikzlibrary{arrows}


\pagestyle{fancy}
\fancyhf{}
\fancyfoot[CE,CO]{Juan Pablo Capurro}
\fancyfoot[LE,RO]{\thepage}

\title{Lab kern0}
\author {Juan Pablo Capurro}

\lstset{showstringspaces=false}

%%%%%%%%%%%%%%%%%%%%%%%%%%%%%%%%%%%%%
% hacer que lstlisting no se corte
%%%%%%%%%%%%%%%%%%%%%%%%%%%%%%%%%%%%%
\usepackage{float}

\floatstyle{plain} % optionally change the style of the new float
\newfloat{Code}{H}{myc}
\lstloadlanguages{make, C}
\lstset{basicstyle=\ttfamily\footnotesize}

%%%%%%%%%%%%%%%%%%%%%%%%%%%%%%%%%%%%%%

%%%% para poder hacer \code{}, obviamente copypasteado de SO%%%%%%%%
\definecolor{light-gray}{gray}{0.95}
\newcommand{\code}[1]{\colorbox{light-gray}{\texttt{#1}}}

\begin{document}
\maketitle

\section{kern0-hlt}
\subsection{comparacion entre loop infinito e instruccion halt}
\begin{tabular}{|l|c|c|}
    \hline \textbf{Version}           & \textbf{Usage} & \textbf{Events/S}\\
    \hline Loop infinito              & 978ms/S        & 229\\
    \hline Instruccion \code{hlt} & 2ms/S          & 60 \\
    \hline
\end{tabular}
\bigskip
\begin{lstlisting}[language=C, gobble=4, tabsize=4]
	#include <stdio.h>
	int main(void){
        int i=0;
        printf("Hello, World!");
        for (i=0; i<1; i++)
            printf("\n");
        return 0;
	}
\end{lstlisting}
\section{kern0-gdb}
\subsection{¿Por qué hace falta el modificador \code{x} al imprimir \code{\%eax}, y no así \code{\%esp}?}
\begin{lstlisting}[gobble=4,tabsize=4]
    (qemu) info registers
    EAX=2badb002 EBX=00009500 ECX=00100000 EDX=00000511
    ESI=00000000 EDI=00102000 EBP=00000000 ESP=00006f0c
    EIP=00100000 EFL=00000006 [-----P-] CPL=0 II=0 A20=1 SMM=0 HLT=0
    ES =0010 00000000 ffffffff 00cf9300 DPL=0 DS   [-WA]
    CS =0008 00000000 ffffffff 00cf9a00 DPL=0 CS32 [-R-]
    SS =0010 00000000 ffffffff 00cf9300 DPL=0 DS   [-WA]
    DS =0010 00000000 ffffffff 00cf9300 DPL=0 DS   [-WA]
    FS =0010 00000000 ffffffff 00cf9300 DPL=0 DS   [-WA]
    GS =0010 00000000 ffffffff 00cf9300 DPL=0 DS   [-WA]
    LDT=0000 00000000 0000ffff 00008200 DPL=0 LDT
    TR =0000 00000000 0000ffff 00008b00 DPL=0 TSS32-busy
    GDT=     000caa68 00000027
    IDT=     00000000 000003ff
    CR0=00000011 CR2=00000000 CR3=00000000 CR4=00000000
    DR0=00000000 DR1=00000000 DR2=00000000 DR3=00000000
    DR6=ffff0ff0 DR7=00000400
    EFER=0000000000000000
    FCW=037f FSW=0000 [ST=0] FTW=00 MXCSR=00001f80
    FPR0=0000000000000000 0000 FPR1=0000000000000000 0000
    FPR2=0000000000000000 0000 FPR3=0000000000000000 0000
    FPR4=0000000000000000 0000 FPR5=0000000000000000 0000
    FPR6=0000000000000000 0000 FPR7=0000000000000000 0000
    XMM00=00000000000000000000000000000000 XMM01=00000000000000000000000000000000
    XMM02=00000000000000000000000000000000 XMM03=00000000000000000000000000000000
    XMM04=00000000000000000000000000000000 XMM05=00000000000000000000000000000000
    XMM06=00000000000000000000000000000000 XMM07=00000000000000000000000000000000
\end{lstlisting}

\end{document}

